\documentclass[11pt]{article}
\usepackage[margin=1in]{geometry}
\usepackage{amsmath,amsthm,amssymb}
\usepackage{enumitem, graphicx, float, caption}
\usepackage{amsmath}
\usepackage{bm}
\usepackage{parskip}
\usepackage{lipsum}
\usepackage{tikz}
\usepackage{listings}
\usepackage{xcolor}
\usepackage{float}
\usepackage{subcaption}
\usepackage{graphicx}
\usepackage{hyperref}
\usepackage[utf8]{inputenc}
\usetikzlibrary{arrows.meta}

\definecolor{codegreen}{rgb}{0,0.6,0}
\definecolor{codegray}{rgb}{0.5,0.5,0.5}
\definecolor{codepurple}{rgb}{0.58,0,0.82}
\definecolor{backcolour}{rgb}{0.95,0.95,0.92}

\lstdefinestyle{mystyle}{
    backgroundcolor=\color{backcolour},
    commentstyle=\color{codegreen},
    keywordstyle=\color{magenta},
    numberstyle=\tiny\color{codegray},
    stringstyle=\color{codepurple},
    basicstyle=\ttfamily\tiny,
    breakatwhitespace=false,
    breaklines=true,
    captionpos=b,
    keepspaces=true,
    numbers=left,
    numbersep=5pt,
    showspaces=false,
    showstringspaces=false,
    showtabs=false,
    tabsize=2
}

\lstset{xleftmargin=.020\textwidth, xrightmargin=.020\textwidth}

\setcounter{MaxMatrixCols}{26}
\setlength\parindent{0pt}
\counterwithin{equation}{enumi}


\begin{document}
    \noindent Nathan Burwig \\
    Math 87 HW 6 \\
    Due 11/02/2022
    
    \hrulefill

    \section{You can't handle the truth!}

    Let’s look at the impact of a juror’s likelihood to vote correctly and the chance of a wrongful
    conviction. A jury consists of 12 members, and each member must vote either guilty or not guilty.
    Assume that the vote of each member is independent Suppose that the probability that a juror votes
    correctly (i.e. in line with the actual innocence/guilt of the person on trial)
    is 70\%. Disclaimer: these numbers are made up, though there are many studies done on this topic.

    \begin{enumerate}
        \item \textbf{What is the probability that there is an inconclusive (tied)
            voting outcome}

            For much of this problem, we will be using the binomial
            distribution. We take the probability of some event then to be...
            \[
                P(X=m)= \begin{pmatrix} n \\ m \end{pmatrix} p^m(1-p)^{n-m}
            \]
            In the case of a tied voting outcome, we know that $m=6$ and $n=6$
            with a p-value of $0.7$. This gives the following outcome...
            \[
                P(X=6)=\begin{pmatrix} 12 \\ 6 \end{pmatrix} (0.7)^6(0.3)^6 = 0.079
            \]
        \item \textbf{If there is an inconclusive verdict, then a judge breaks the tie. 
            Suppose a judge votes correctly 80\% of the time. What is the 
            probability of a tie and a judge voting correctly?}

            The probability of this happening is a compound probability. We
            know that there is a 0.079 probability of the first event occuring,
            so that, combined with the following probability associated with
            the judge voting correctly...
            \[
                P(X=6)\times 0.8 = 0.079 \times 0.8 = 0.063
            \]
            I don't entirely know if I'm doing this correctly (I've always been
            bad at probability) but if this is correct, these numbers are
            scary.
        \item \textbf{A verdict is reached if at least 7 jury members vote the same way, 
            or a tie is broken in the manner given above. What is the 
            probability that the correct verdict is reached?}

            So we know we can reach the correct verdict a few different ways.
            Either a) it is reached after a tie, followed by the correct
            verdict by the judge, or b) anywhere between 7-12 jurors vote
            correctly. This can be represented the following way.
            \[
                P(X=6) * 0.8 + \sum^{12}_{m=7} \begin{pmatrix} 12 \\ m \end{pmatrix}
                (0.7)^m(0.3)^{12-m}
            \] 
            Which, when evaluated results in the value $0.9455$ or $94.55\%$
        \item \textbf{Of the times where the jury reaches the incorrect verdict, 
            suppose 90\% of people on trial are guilty but not convicted, 
            and 10\% are innocent but convicted. What is the probability of a
            wrongful conviction?}
            
            It is given that an innocent person is convicted only 10\% of the
            time. We know that the probability of an incorrect verdict is 
            $1-0.9445$ or $0.0545$, so that means the probability of an
            innocent conviction is $0.0545 * 0.10 = 0.00545$ or $0.545\%$.
        \item \textbf{What is the expected number of people wrongfully
            convicted in 2 million trials?}

            We get this number simply by multiplying 2 million by the number
            percentage chance of a wrongful conviction which we found in the
            previous part of this problem. We get $2,000,000 * 0.00545 =
            10,900$ so approximatly $11000$ people will be wrongly
            convicted out of $2$ million.
        \item \textbf{Write a \texttt{python} function that takes in the likelihood of 
            a juror to vote correctly and returns the probability of a wrongful 
            conviction. You may use \texttt{scipy.stats.binom} or compute your own 
            binomial distribution. Include your code in your report.}

            Below is the code I used to calculate this for the given values.

            \begin{lstlisting}[style=mystyle, linewidth=0.94\linewidth, 
                                language=Python, gobble=8, caption=The Program]
                from scipy.stats import binom

                ## a function to determine wrongful conviction rates
                def wrongful_conviction(p):
                    ## initial variables
                    prob = 0
                    a = range(7, 13)
                
                    ## loop through to get probabilities associated with problem
                    for i in a:
                      prob = prob + binom.pmf(12-i, 12, p)
                    
                    ## the probability there is a tie times probability judge 
                    ## gives incorrect verdict
                    tie = binom.pmf(6, 12, p) *.2
                
                    ## sum the two associated probabilites
                    prob = (tie + prob) * .1
                
                    return(prob)
            \end{lstlisting}

        \item \textbf{Test your function on likelihoods 50\%, 60\%, 70\%, 80\%, and 
            90\%. Report both the probability of a wrongful conviction, 
            and the expected number of people wrongfully convicted in 2 
            million trials. Discuss your results.}

            When I run my code, I get the following output.

            \begin{center}
                \begin{tabular}{|| c | c | c ||}
                    \hline
                    P-Correct Verdict   &   P-Wrongful Conv.    &   \# Wrongful Conv. \\
                    \hline\hline
                    0.5 	&	 0.0432 	&	 86464 	\\
                    0.6 	&	 0.0194 	&	 38705 	\\
                    0.7 	&	 0.0054 	&	 10890 	\\
                    0.8 	&	 0.0007 	&	 1400 	\\
                    0.9 	&	 1.484e-5 	&	 29 	\\
                    \hline
                \end{tabular}
            \end{center}
        \item \textbf{What are some flaws in the assumptions of the problem?}

            The most blatant flaws to me are the assumptions about the
            probability of a correct verdict. The ability to judge this depends
            on the judge, the trial, the defense, the prosecution, all sorts of
            variables that are very difficult to keep track of. Trying to
            reduce the complexity of a court room to simple probabilities is
            bound to be innacurate. 

            I also feel like it is very difficult to actually come up with
            accurate probabilities for wrongfully convicted persons. There
            isn't really a way to know how many people have been wrongfully
            convicted, so tryign to ascribe probabilities to these types of
            things is also bound to be innacurate.

    \end{enumerate}

    \section{Geometry feat. Monte Carlo}

    Let $C$ be the circle defined by $(x-1.25)^2 + (y-1.25)^2=0.4$. We know it
    has interior area of $.4\pi$. Assume we don't know this, and use Monte
    Carlo Rejection Sampling to determine the area of this circle.

    \begin{enumerate}
        \item \textbf{Run this experiment with sample count (1, 5, 10, 50, 100, 1000, 
            5000, 10000, 50000, 100000, 500000, 1000000).  Compute the error 
            (relative the true area) both as a value and as a percentage}

            In order to run this experiment, there are a couple quantities that
            we would like to know. Namely, we want to know what equation we can
            use to find the area of the circle. We can do this by taking the
            number of points that land within the circle, and comparing it to
            the total number of points within the square. This ratio will give
            us the percentage area taken up in the square, so if we multiply by
            the area of the given square, we know the area of the circle within
            the region.

            We also want to know the actual area (given) and the radius ($\sqrt{.4}$)
            along with how to calculate error. Our equations of interest, then,
            are the following...
            \begin{align*}
                \text{Area} &= \text{Area of Square} \cdot \frac{\text{\# of 
                points within circle}}{\text{\# of points in total}} \\
                \text{Error} &= \frac{\text{calculated - expected}}
                {\text{expected}} 
            \end{align*}
            With all of this in mind, we can begin working on the actual
            problem. We simulate points in our system according to the layour
            I've given above, and "reject" any points that are outside the area
            given by the formula for our circle. The region we are simulating
            in is the square from $0 \leq x \leq 2$ and $0 \leq y \leq 2$

            I'll give the code I used to do this in case the above
            explanation was unclear.
            \newpage

            \begin{lstlisting}[style=mystyle, linewidth=0.94\linewidth, 
                                language=Python, gobble=12, caption=Monte Carlo
                                Rejection Sampling for Area of Circle]
                import random, math

                ## iter = number of iterations (integer only)
                ## R    = radius of circle (integer or otherwise)
                ## ran  = range of region to check (ie [0, 2] -- only 2D range accepted)
                def circle(iter, R, ran):
                    total, in_circle = 0, 0
                
                    for i in range(iter):
                        ## generate random points uniformly over the given interval 'ran'
                        x = random.uniform(ran[0], ran[1])
                        y = random.uniform(ran[0], ran[1])
                
                        ## sample rejection: remove points not within given area
                        if (x - 1.25) ** 2 + (y - 1.25) ** 2 <= R ** 2:
                            in_circle += 1
                        total += 1
                
                    ## calculate the area via ratio of area to volume and expected area
                    # area is area of square times ratio of dots in circle vs out
                    area = (ran[1] ** 2) * in_circle / total
                
                    # expected area is just area of the circle of radius R
                    expected = math.pi * R ** 2
                
                    ## check for errors (on small samples no point lands in 
                    ## region so area = 0 causing divide by 0)
                    if(area != 0):
                        error = abs((area-expected)/expected)
                    else: 
                        error = "err, div by 0"
                
                    ## print relevant info
                    print("Given", iter, R, ran)
                    print("  expected:\t", expected)
                    print("  calculated:\t", area)
                    print("  error:\t", error)
            \end{lstlisting}
            Using the above code we can generate as many points in the region
            we want. When we do this, we get the following for the given
            samples.
            \begin{center}
                \begin{tabular}{|| c || c | c | c ||}
                    \hline
                    Samples &   Calculated  &   Expected    &   Error   \\
                    \hline
                    \hline
                    1 	    &	 4.0 	    &	 1.2566 	&	 2.18309    \\
                    5 	    &	 1.6 	    &	 1.2566 	&	 0.27323    \\
                    10 	    &	 1.2 	    &	 1.2566 	&	 0.04507    \\
                    50 	    &	 1.68 	    &	 1.2566 	&	 0.33690    \\
                    100 	&	 1.08 	    &	 1.2566 	&	 0.14056    \\
                    1000 	&	 1.316 	    &	 1.2566 	&	 0.04723    \\
                    5000 	&	 1.2568 	&	 1.2566 	&	 0.00012    \\
                    10000 	&	 1.2492 	&	 1.2566 	&	 0.00591    \\
                    50000 	&	 1.25256 	&	 1.2566 	&	 0.00324    \\
                    100000 	&	 1.26056 	&	 1.2566 	&	 0.00312    \\
                    500000 	&	 1.258504 	&	 1.2566 	&	 0.00148    \\
                    1000000 &	 1.2518 	&	 1.2566 	&    0.00384    \\
                    \hline
                \end{tabular}
            \end{center}
            So we can see that we get fairly close as we get larger and larger
            in our sample count. I find it interesting that we only get
            \textit{so} close. The last several steps only seemed to increase
            the accuracy by a little, and even then, it seemed largely random.

            Also, it is worth noting the \texttt{err}. When you have such a
            small number of samples, it becomes increasingly likely that the
            one or five points you cast onto your system will actually land
            within the circle, so your area becomes zero which causes a divide
            by zero error. It took several attempts of running this before it
            gave a non-error result for 5 and 10 samples.
            \newpage

        \item \textbf{Let $C'$ be the cirlce defined by $(x-.75)^2 + (y-.75)^2
            = .3$. Find the area of the union of these two points using a
            modified version of the above algorithm}

            We know that we want to find the area contained between the two
            circles. The first issue here is that I don't know how to calculate
            that normally, so I don't actually have anything to compare
            against. Looking up some code online I was able to find (\url{
            https://scipython.com/book/chapter-8-scipy/problems/p84/overlapping-circles/})
            as well as some help from wolfram, I was able to find the
            \textit{intersection} area of the two circles, and I know that the
            combined area will just be the addition of the two individual areas
            minus the intersection area)

            We will use an almost identical version of the code I have shown
            above, apart from two lines. Our rejection line (points not within
            the circle) now specifies whether is within either of the two
            circles. Our \texttt{expected} is now the sum of the two areas
            minus the intersection as calculated by wolfram and the code given
            in the provided link.

            With this, we get the following...
            \begin{center}
                \begin{tabular}{|| c || c | c | c ||}
                    \hline
                    Samples &   Calculated  &   Expected    &   Error   \\
                    \hline
                    \hline
                    1000000 &   1.8867396   &   1.888137    &   0.00039 \\
                    \hline
                    \hline
                \end{tabular}
            \end{center}
            We know the sum of the two areas to be $2.19911$, so having a value
            of $\approx 1.88$ makes sense as it is less than the total area,
            and is within a very small error of the actual value.
    \end{enumerate}

\end{document}

